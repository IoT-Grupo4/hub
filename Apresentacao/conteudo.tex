
\begin{frame}
\titlepage % Print the title page as the first slide
\end{frame}

% \begin{frame}
% \frametitle{Agenda} % Table of contents slide, comment this block out to remove it
% \tableofcontents % Throughout your presentation, if you choose to use \section{} and \subsection{} commands, these will automatically be printed on this slide as an overview of your presentation
% \end{frame}

%----------------------------------------------------------------------------------------
%   PRESENTATION SLIDES
%----------------------------------------------------------------------------------------

%------------------------------------------------
\begin{frame}
% \section{Objetivo do Projeto} % Sections can be created in order to organize your presentation into discrete blocks, all sections and subsections are automatically printed in the table of contents as an overview of the talk
%------------------------------------------------
\frametitle{Objetivo do Projeto}

\begin{itemize}
	\item Desenvolver sistema de roteamento orientado a contexto capaz de:
	\begin{itemize}
		\item Perceber, inferir e aprender com as informações obtidas por sensores de um certo ambiente/contexto.
		\item Selecionar, com isso, a melhor rota para envio de dados.
		\item Demonstrar conceitos de sensibilidade ao contexto para criação de dispositivos inteligentes.
	\end{itemize}
\end{itemize}
\end{frame}

% \subsection{Subsection Example} % A subsection can be created just before a set of slides with a common theme to further break down your presentation into chunks

\begin{frame}
\frametitle{Arquitetura}
\begin{description}
	\item [Linguagens:] \begin{itemize}
		\item Python
		\item Shell (Bash)
	\end{itemize}
	\item [Camada 2:]
	\begin{itemize}
		\item Ethernet (802.3)
	\end{itemize}
	\item [Roteamento:]
	\begin{itemize}
		\item OSPF no Quagga
	\end{itemize}
	\item [Hardware:]
	\begin{itemize}
		\item Raspberry Pi 3 (Raspbian/Debian Linux)
	\end{itemize}
\end{description}

\end{frame}

%------------------------------------------------

\begin{frame}
	\frametitle{Camadas do sistema}

	\begin{figure}[h]
		\centering
		\includegraphics[width=0.6\textwidth]{"../Relatorio/Artigo IoT-G4/figs/system-layer"}
		% \caption{System Layers}
		\label{System-Layers}
 	\end{figure}

\end{frame}

%------------------------------------------------

\begin{frame}
	\frametitle{Topologia}

	\begin{figure}[h]
		\centering
		\includegraphics[width=0.9\textwidth]{"../Relatorio/Artigo IoT-G4/figs/topologia-2"}
		% \caption{System Layers}
		\label{topologia}
 	\end{figure}

\end{frame}

%------------------------------------------------

\begin{frame}
	\frametitle{Ontologia}
	Separada em três entidades:
	\begin{itemize}
		\item Ambiente
		\item Microcontrolador
		\item Sensor
	\end{itemize}

	\begin{figure}[h]
		\centering
		\includegraphics[width=0.9\textwidth]{"../Relatorio/Artigo IoT-G4/figs/Ambiente"}
		% \caption{System Layers}
		\label{topologia}
 	\end{figure}

\end{frame}

%------------------------------------------------

\begin{frame}
	\frametitle{Ontologia}
	
	\begin{figure}[h]
		\centering
		\includegraphics[width=0.9\textwidth]{"../Relatorio/Artigo IoT-G4/figs/Microcontrolador"}
		% \caption{System Layers}
		\label{topologia}
 	\end{figure}

\end{frame}

%------------------------------------------------

\begin{frame}
	\frametitle{Ontologia}
	
	\begin{figure}[h]
		\centering
		\includegraphics[scale=0.4]{"../Relatorio/Artigo IoT-G4/figs/Sensor"}
		% \caption{System Layers}
		\label{topologia}
 	\end{figure}

\end{frame}

%------------------------------------------------

\begin{frame}
	\frametitle{Inferência}
	Questões de inferência:
	
	\begin{itemize}
		\item Quais as rotas entre os dispositivos?
		\item Quais as características do ambiente?
		\item Qual a melhor rota entre os dispositivos?
		\item Quem é mais confiável?
		\item Quais são as rotas alternativas?
	\end{itemize}
\end{frame}

%------------------------------------------------

\begin{frame}
	\frametitle{Sensores} % inserir figura dos sensores

\begin{description}
	\item [DHT11] Umidade e Temperatura \includegraphics[width=0.1\textwidth]{"../Relatorio/Artigo IoT-G4/figs/dht11"}
	\includegraphics[width=0.1\textwidth]{"../Relatorio/Artigo IoT-G4/figs/dht11-pinagem"}
	\begin{itemize}
		\item Leituras de temperaturas entre 0 a 50$^o$ Celsius e umidade entre 20 a 90\%.
	\end{itemize}
	\item [LDR] Luminosidade \includegraphics[width=0.1\textwidth]{"../Relatorio/Artigo IoT-G4/figs/ldr"}
	\begin{itemize}
		\item O LDR(\textit{Light Dependent Resistor}) possui uma característica que faz com que sua resistência varie conforme a luminosidade incendida  sobre ele.
	\end{itemize}
\end{description}

\end{frame}

%------------------------------------------------


\begin{frame}
	\frametitle{Sensores - Circuito}

\begin{figure}[!tb]
\begin{center}\begin{circuitikz}[scale=1]
  \draw (0,0) -- (0,2) to [R=10<\kilo\ohm>,*-*] (2,2);
  \draw (0,2) -- (0,3) to [R=10<\kilo\ohm>,*-*] (2,3);
  \draw (0,3) to[short,*-o] (0,5) node[above]{$V_{CC}=3.3V$}; % Power supply
  \draw (0,4) to[short,*-] (1.9,4) -- (1.9,4.5);
  \draw (0,0) to [phR=LDR,*-*](2,0) to [eC,l=10<\micro\farad>,*-*](4,0) -- (4,4) -- (2.2,4) -- (2.2,4.5);
  \draw (4,3.5) to[short,*-] (4.5,3.5) node[ground]{};
  \draw (2,2) -- (2,4.5);
  \draw (2.1,4) -- (2.1,4.5);
  \draw (1.4,5.7) rectangle (2.7,4.5)
    node at(2.05,5.1){DHT11};
   %gpio pins
  \draw (2,0) to[short,*-o] (2,-1) node[right]{Pin11};
  \draw (2,2) to[short,*-o] (2.5,2) node[right]{Pin7};

 \end{circuitikz} \end{center}
% \caption{Sensor circuit}
\label{sensor-circuit}
\end{figure}

\end{frame}

%------------------------------------------------

\begin{frame}
	\frametitle{Funcionamento}

	\begin{figure}[h]
		\centering
		\includegraphics[width=0.6\textwidth]{"../Relatorio/Artigo IoT-G4/figs/sequencia"}
		% \caption{System Layers}
		\label{sequencia}
 	\end{figure}
\end{frame}

%------------------------------------------------

\begin{frame}[t,fragile]
	\frametitle{Dados Coletados - Sensores} % inserir gráficos do CSV
\begin{figure}
	% \centering

   \begin{tikzpicture}[baseline=0pt]
    \begin{axis}[
         width=\linewidth*0.45, % Scale the plot to \linewidth
         grid=major,
         clip=false, % avoids clipping of plot lines at the axis border
         ylabel=Light,
       ]
       \addplot[mark=none]
        table[x=x,y=light,col sep=comma] {../Relatorio/externos/dados/LDR.csv};
     \end{axis}
   \end{tikzpicture}
%
   \begin{tikzpicture}[baseline=0pt]
	\begin{axis}[
         width=\linewidth*0.45, % Scale the plot to \linewidth
         grid=major,
         clip=false, % avoids clipping of plot lines at the axis border
         ylabel=Temperature,
       ]
       \addplot[mark=none]
       table[x=x,y=temperature,col sep=comma] {../Relatorio/externos/dados/DHT.csv};
     \end{axis}
   \end{tikzpicture}
%
   \begin{tikzpicture}[baseline=0pt]
     \begin{axis}[
         width=\linewidth*0.45, % Scale the plot to \linewidth
         grid=major,
         clip=false, % avoids clipping of plot lines at the axis border
         ylabel=Humidity,
       ]
       \addplot[mark=none]
       table[x=x,y=humidity,col sep=comma] {../Relatorio/externos/dados/DHT.csv};
     \end{axis}
   \end{tikzpicture}
   % \caption{Humidity read from sensor}
\end{figure}

\end{frame}

%------------------------------------------------

\begin{frame}
	\frametitle{Resultados Experimentais}

	\begin{figure}[h]
		\centering
		\includegraphics[width=0.9\textwidth]{"../Relatorio/Artigo IoT-G4/figs/resultado1"}
		% \caption{System Layers}
		\label{resultado1}
 	\end{figure}
\end{frame}

%------------------------------------------------

\begin{frame}
	\frametitle{Resultados Experimentais}

	\begin{figure}[h]
		\centering
		\includegraphics[width=0.9\textwidth]{"../Relatorio/Artigo IoT-G4/figs/resultado2"}
		% \caption{System Layers}
		\label{resultado2}
 	\end{figure}
\end{frame}

%------------------------------------------------

\begin{frame}
	\frametitle{Resultados Experimentais}

	\begin{figure}[h]
		\centering
		\includegraphics[width=0.9\textwidth]{"../Relatorio/Artigo IoT-G4/figs/resultado3"}
		% \caption{System Layers}
		\label{resultado3}
 	\end{figure}
\end{frame}

%------------------------------------------------

