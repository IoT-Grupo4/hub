\documentclass[conference]{IEEEtran}


% Inclusão de pacotes
\usepackage[utf8]{inputenc}
\usepackage{cite}
\usepackage{amsmath}
\usepackage{algorithmic}
\usepackage{array}
\usepackage{url}

\usepackage{tikz} % diagramas e UML
\usepackage[siunitx]{circuitikz} % circuitos
\usepackage{lipsum} % encher linguiça. em latin
\usepackage{listings} % códigos bonitos *.*
\usepackage{color, colortbl}
\usepackage{parallel}
\usepackage{amssymb}
\usepackage{dblfloatfix} % magica com imagens no IEEE
\usepackage{graphicx}
\usepackage{subcaption}

\usepackage[nameinlink]{cleveref}
\usepackage{longtable}
\usepackage{mathtools}
\usepackage{multicol}
\usepackage{pgfplots}
\usepackage{csvsimple}
\usepackage{hyperref}

% Configuração de pacotes

%  Configurações com base em: https://gitlab.com/Ziul/tcc

\newcommand{\code}{\small\ttfamily}

%Definições para código com fundo listrado
\newcommand\realnumberstyle[1]{#1}
\makeatletter
\newcommand{\zebra}[3]{%
    {\realnumberstyle{#3}}%
    \begingroup
    \lst@basicstyle
    \ifodd\value{lstnumber}%
        \color{#1}%
    \else
        \color{#2}%
    \fi
        \rlap{
        \color@block{0.94\textwidth}{\ht\strutbox}{\dp\strutbox}%
        \hspace*{\lst@numbersep}%
        }%
    \endgroup
}
\makeatother

\lstset{%
language=C++,           %linguagem
numbers=left,           %posição dos números
stepnumber=1,           %frequencia de aparição dos números
numbersep=5pt,
numberstyle=\zebra{gray!15}{white!35},
basewidth={0.6em,0.45em},
fontadjust=true,
mathescape=true,
tabsize=4,
commentstyle=\color{blue},
%
literate={á}{{\'a}}1
{à}{{\`a}}1
{ã}{{\~a}}1
{é}{{\'e}}1
{É}{{\'E}}1
{ê}{{\^e}}1
{õ}{{\~o}}1
{í}{{\'i}}1
{ó}{{\'o}}1
{ú}{{\'u}}1
{&}{{\&}}1
{ç}{{\c c}}1
{³}{{$^3$}}1
{Ω}{{$\Omega$}}1,
%
breaklines=true,
showstringspaces=false,
stringstyle=\color{cyan},
basicstyle=\small\ttfamily}

\lstset{%
language=Python,           %linguagem
% literate={_}{{\_}}1 ,
numbers=left,           %posição dos números
% numbersep=5pt,
numberstyle=\zebra{gray!15}{white!35},
basewidth={0.6em,0.45em},
fontadjust=true,
basicstyle=\footnotesize\ttfamily}


\pgfplotsset{compat=newest}
\usepgfplotslibrary{units} 
