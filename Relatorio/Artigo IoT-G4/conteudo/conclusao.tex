\section{Conclusion}

This article described the possibilities involved in Context-Aware Routing and provided a real application for Internet of Things. The use of sensor networks is adapted to provide data that is used both for external purposes to the network and for purposes of contextualizing the network itself, which redefines the possibilities of data traffic in IoT systems. Previous simulations were implemented and performed in order to assure the proper operation of the firmware and finally, tests in a real platform have been successfully implemented and performed. The main contributions derived here provide a new approach to data routing by improving the use of all available information to increase intelligence in network traffic.

Some practical and theoretical issues remain to be implemented in future works. In addition of choosing more suitable thresholds used to make decisions, it is still necessary to evaluate the possibility of interfering with OSPF routing through Python codes without the need for a centralized control server.

% use section* for acknowledgment
\section*{Acknowledgment}
The authors would like to thank Professors Rafael Timóteo de Sousa Júnior, Caio Cesar and Fábio Lúcio Lopes de Mendonça for their valuable funding and support of this work. The authors also like to extend the acknowledgment to the Department of Electrical Engineering of Brasilia University for providing resources to the development of this work.

